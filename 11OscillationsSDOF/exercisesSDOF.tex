\documentclass[11pt, english, a4paper, twopage]{article}
% Version en 2024 Víctor Bettachini < vbettachini@unlam.edu.ar >

\usepackage[T1]{fontenc}
\usepackage[utf8]{inputenc}

% \usepackage[spanish, es-tabla]{babel}
% \def\spanishoptions{argentina} % Was macht dass?
% \usepackage{babelbib}
% \selectbiblanguage{spanish}
% \addto\shorthandsspanish{\spanishdeactivate{~<>}}

\usepackage{graphicx}
\graphicspath{{../figuresLaTeX/}}
% \usepackage{float}

\usepackage[arrowdel]{physics}
\newcommand{\pvec}[1]{\vec{#1}\mkern2mu\vphantom{#1}}
% \usepackage{units}
\usepackage[separate-uncertainty= true, multi-part-units= single, range-units= single, range-phrase= {~a~}, locale= FR]{siunitx}
\usepackage{isotope} % $\isotope[A][Z]{X}\to\isotope[A-4][Z-2]{Y}+\isotope[4][2]{\alpha}

\usepackage{tasks}
\usepackage[inline]{enumitem}
% \usepackage{enumerate}

\usepackage{hyperref}

% \usepackage{amsmath}
% \usepackage{amstext}
% \usepackage{amssymb}

\usepackage{tikz}
\usepackage{tikz-3dplot}
\usepackage{tikz-dimline}
\usetikzlibrary{calc}
% \usetikzlibrary{math}
\usetikzlibrary{arrows.meta}
\usetikzlibrary{snakes}
\usetikzlibrary{decorations}
\usetikzlibrary{decorations.pathmorphing}
\usetikzlibrary{patterns}

\usepackage[hmargin=1cm,vmargin=3cm, top= 0.75cm,nohead]{geometry}

\usepackage{lastpage}
\usepackage{fancyhdr}
\pagestyle{fancyplain}
\fancyhf{}
\setlength\headheight{28.7pt} 
\fancyhead[LE, LO]{\textbf{Computational Analytical Mechanics} }
% \fancyhead[LE, LO]{\textbf{Mecánica General} }
\fancyhead[RE, RO]{\href{https://ingenieria.unlam.edu.ar/}{$\vcenter{\hbox{\includegraphics[height=1cm]{ambos.pdf}}}$}}
\fancyfoot{\href{https://creativecommons.org/licenses/by-nc-sa/4.0/}{$\vcenter{\hbox{\includegraphics[height=0.4cm]{by-nc-sa_80x15.pdf}}}$} \href{https://ingenieria.unlam.edu.ar/}{DIIT - UNLaM}}
\fancyfoot[C]{ {\tiny Updated \today} }
\fancyfoot[RO, LE]{Page \thepage/\pageref{LastPage}}
\renewcommand{\headrulewidth}{0pt}
\renewcommand{\footrulewidth}{0pt}

\begin{document}
\begin{center}
 \textsc{\large Vibrations | Single degree of freedom}
\end{center}
\begin{enumerate}
\item
\begin{minipage}[t][4cm]{0.65\textwidth}
 \textbf{Constrained pendulum}
 The system shown in the figure consists of a mass $m$ and a rigid bar of length $l$, whose mass is neglected.
 The system is constrained by two springs with stiffness coefficients $k_1$ and $k_2$.
 Obtain the dynamics equation assuming small oscillations and the natural frequency of oscillation of the system.
\end{minipage}
\begin{minipage}[c][3cm][t]{0.3\textwidth}
 \includegraphics[width=\textwidth]{figures/shabana_fig_P1_1.png}
\end{minipage}
\item
\begin{minipage}[t][6cm]{0.65\textwidth}
 \textbf{Audi TT Coupé}
 The figure shows the location of the shock absorbers of an \emph{Audi TT Coupé}.
 The \href{https://media.audiusa.com/assets/documents/original/8253-FINAL2021TTTechSpecs.pdf}{specifications of this machine} indicate that with one passenger and 90\% fuel load it has a weight of \SI{1450}{\kilo\gram}.
 Use the quarter car model, in which it is assumed that each shock absorber supports one quarter of the weight.
 Further simplify this model by eliminating the tire, both its mass and its ability to operate as a shock absorber, to assign this latter task solely to the suspension.
 Since driving in the overdamped regime is uncomfortable, as after a pothole a violent rebound can occur, you must adjust the suspension accordingly.
 From a \href{https://journals.sagepub.com/doi/pdf/10.1177/1687814016648638}{paper in \emph{Advances in Mechanical Engineering}} we take a standard value of $k_s = \SI{12500}{\newton\per\metre}$ for the shock absorber.
\end{minipage}
\begin{minipage}[c][0cm][t]{0.3\textwidth}
 \includegraphics[width=\textwidth]{figures/amortiguadores_AudiCoupeTT}
\end{minipage}
\item
 \begin{minipage}[t][3.5cm]{0.75\textwidth}
 \textbf{Unbalanced motor}
 A \SI{15}{\kilo\gram} electric motor has an unbalanced load of \SI{20}{\gram} at \SI{125}{\milli\metre} from its axis.
 It is bolted to a support that limits its movement to the vertical direction.
 To mitigate its vibration it is damped by four springs of \SI{40}{\kilo\newton\per\metre}, and an oil damper with a coefficient linearly proportional to velocity adjusted so that $c = 0.4 C_c$ ($C_c$, critical damping coefficient).
 Obtain an approximate range of motor operating frequencies where the vibration is less than \SI{0.2}{\milli\metre}.
\end{minipage}
\begin{minipage}[c][2cm][t]{0.2\textwidth}
 \includegraphics[width=\textwidth]{figures/beer_fig_P19_144}
\end{minipage}
\item
\begin{minipage}[t][3.5cm]{0.75\textwidth}
 \textbf{Cam}
 Cams are characterized by displacement maps, which \emph{map} the function of their radius to a linear displacement of the follower.
 The figure shows how a heart-shaped cam allows the displacement of the latter to increase and decrease linearly from a peak.
 Assume that at the peak the displacement is \SI{5}{\centi\metre} and at the minimum it is zero, and that a motor makes the cam rotate at $6\,\mathrm{rpm}$.
 Assuming that this system is used to force the system from the example given in class, plot the displacement of $m$ in steady state as a function of time during four rotations of the cam.
\end{minipage}
\begin{minipage}[c][2cm][t]{0.2\textwidth}
 \includegraphics[width=\textwidth]{figures/nok_wikkelmachine}
\end{minipage}
\end{enumerate}
\end{document}