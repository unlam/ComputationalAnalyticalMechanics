\documentclass[11pt, a4paper, twoside]{article}

\input{../figuresLaTeX/introUNLaM}


\begin{document}

\begin{center}
  \textsc{\large Generalized coordinates | Constraints | Kinetic and potential energies}
\end{center}
\noindent
Exercises marked with (*) have extra difficulty, don't hesitate to ask for help.

\begin{enumerate}

\item
	\begin{minipage}[t][5cm]{0.7\textwidth}
		\textbf{Pendulum with free point of support} [Landau \S5 ex. 2]\\
		Particle of mass $m_2$ is hanging from a rigid bar of length $\ell$ and negligible mass. On the other end there is a device of mass $m_1$ linked to a horizontal bar, and it's free to move horizontally along the $x$ axis. The device allows the hanging bar to span any angle $\varphi$ respect to the vertical axis. 
		\begin{enumerate}
			\item Write expressions for kinetic energy, $T$ and potential, $V$, as functions of the generalized coordinates suggested by the figure.
			\item Verify that if you fix the position of mass $m_1$ you recover the expressions of $T$ and $V$ of an ideal pendulum.
		\end{enumerate}
	\end{minipage}
	\begin{minipage}[c][1cm][t]{0.3\textwidth}
		\includegraphics[width=0.75\textwidth]{figures/landauS52_fig2.png}
	\end{minipage}



\item
	\begin{minipage}[t][4cm]{0.7\textwidth}
		\textbf{Coplanar double pendulum} [Landau \S5 ex. 1]\\
		A ridig bar of lentgh $\ell_1$ of negligible mass has a particle of mass $m_1$ attached to one end. There is a second bar of negligible mass hanging from the first one, of length $\ell_2$, with a particle of mass $m_2$ attached to the other end too.
		\begin{enumerate}
			\item Write expressions for kinetic energy, $T$ and potential, $V$, as functions of the generalized coordinates suggested by the figure.
			\item Verify that you recover the expressions of $T$ and $V$ of an ideal pendulum if you set $m_1=0$, $\varphi_1 = \varphi_2 = \varphi$ and  $\ell_1 = \ell_2 = \frac{\ell}{2}$.
		\end{enumerate}
	\end{minipage}
	\begin{minipage}[c][0.5cm][t]{0.3\textwidth}
		\includegraphics[width=0.75\textwidth]{figures/landauS52_fig1.png}
	\end{minipage}



\item
	\begin{minipage}[t][7.1cm]{0.5\textwidth}
		(*) \textbf{Pendulum with rotating point of support} [Marion (e) ex. 7.5] [Landau \S5 ex. 3]\\
		A particle of mass $m$ is attached to the end of a rigid bar of length $b$. The point of support is linked to a vertical circle of radius $a$ and it rotates with constant frequency $\omega$. It is assumed that all positions lie in the same plane and the mass of the bar is negligible. Calculate the kinetic energy, $T$, and potential $V$, of the particle of mass $m$.
	\end{minipage}
	\begin{minipage}[c][3cm][t]{0.5\textwidth}
		\includegraphics[width=0.75\textwidth]{figures/marion_fig7_3.png}
	\end{minipage}



\item
	\begin{minipage}[t][4.5cm]{0.65\textwidth}
		(*) \textbf{Coupled weights rotating about a vertical axis} [Landau \S5 ex. 4]\\
		Particle with mass $m_2$ moves on a vertical axis and the whole system rotates about this axis with a constant angular velocity $\Omega$. This particle is linked to two particles of mass $m_1$ through bars of length $a$ and negligible mass, and at the same time these particles are linked to the fixed point $A$ trough identical bars, forming the variable angle $\theta$ respect to the vertical axis.
		Calculate the kinetic energy of each of the three particles and find a compact expression of the kinetic energy of the whole system.
	\end{minipage}
	\begin{minipage}[c][1cm][t]{0.35\textwidth}
		\includegraphics[width=0.75\textwidth]{figures/landauS52_fig4.png}
	\end{minipage}


\end{enumerate}
\end{document}
