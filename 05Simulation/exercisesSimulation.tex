\documentclass[11pt, a4paper, twoside]{article}

\input{../figuresLaTeX/introUNLaM}


\begin{document}
\begin{center}
  % \textsc{\large Mecánica general}\\
  \textsc{\large Simulation | Numerical solutions for the Euler-Lagrange equation}
\end{center}

\noindent
In the following exercises, you will solve numerically the Euler-Lagrange equation for each generalized coordinate. Plotting these solutions, using the given initial conditions and within the given time ranges, you will be simulating the dynamics of these systems.\\
Use \(|\vec{g}| = \SI{9.81}{\metre\per\second\squared}\) for the magnitude of the acceleration due to gravity.\\
Exercises marked with (*) have extra difficulty, don't hesitate to ask for help.


\begin{enumerate}


\item 
\begin{minipage}[t][2.5cm]{0.7\textwidth}
\textbf{Atwood machine}\\
Time from \(t = \SI{0}{\second}\) to \(t = \SI{10}{\second}\).
Parameters and initial conditions:\\
\(\ell_\mathrm{rope} > \SI{150}{\metre}\), 
\(R_{\mathrm{pulley}\,1} = \SI{0.5}{\metre}\), \\ 
\(m_1 = \SI{8}{\kilo\gram}\), 
\(m_2 = \SI{1}{\kilo\gram}\), 
\(M_\mathrm{pulley} = \SI{4}{\kilo\gram}\), \\
\(x(t=0) = \SI{25}{\metre}\), 
\(\dot{x}(t=0) = -\SI{10}{\metre\per\second}\).
\end{minipage}
\begin{minipage}[c][2cm][t]{0.3\textwidth}
	\hspace{0.5cm}
	\includegraphics[width=0.6\textwidth]{figures/marion_fig2_1a}
\end{minipage}



\item
	a) \textbf{Ideal pendulum} [Marion ex. 7.2] \\
	b) \textbf{Pendulum with free support} [Landau \S5 ex. 2]\\ 
	c) \textbf{Double pendulum} [Landau \S5 ex. 1] 
\begin{tasks}(3)
	\task \begin{tikzpicture}[scale= 1.0]
  	\draw [arrows=-latex] (-1,2) -> (-1,1) node [above=15, right=2] {\(\vec{g}\)}; % g vertical
		\draw [ultra thick] (-1.5,3) -- (2,3);
		\fill [pattern = north east lines] (-1.5,3) rectangle (2,3.2); % techo
		\draw [dashed] (0,3) -- (0,-.25);	% vertical
		\draw [thick] (0,3) -- +(-60:3) node[midway,above,right=2] {\(\ell\)};	% inclinada +:relativa, -60 degrees, length 3
		\shade [ball color=black!80] ($(0,3)+(-60:3)$) circle(0.25) node [] {\color{white} $m$};
    \draw [arrows=-latex] (0,.4) -> (1.25,.4) node [midway, above] {\( \psi \)}; % desplazamiento horizontal
		\draw [arrows=-latex] (0,0) arc [start angle=-90, end angle=-65, radius=3] node [below=12, left=8] {\( \varphi \)};
	\end{tikzpicture}
	\task \includegraphics[height=0.2\textwidth]{figures/landauS52_fig2}
	\task \includegraphics[height=0.2\textwidth]{figures/landauS52_fig1}
\end{tasks}
Time from \(t = \SI{0}{\second}\) to \(t = \SI{10}{\second}\). Parameters and initial conditions:
\begin{enumerate}
	\item \(m = \SI{3}{\kilo\gram}\), 
				\(\ell = \SI{2}{\metre}\), 
				\(\varphi (t=0) = \frac{\pi}{4}\), \(\dot{\varphi} (t=0) = 0\).
	\item \(m_1 = \SI{3}{\kilo\gram}\), \(m_2 = \SI{1}{\kilo\gram}\),   
				\(\ell = \SI{2}{\metre}\), 
				\(x(t=0) = \SI{1}{\metre}\), \(\dot{x} (t=0) = \SI{0.5}{\metre\per\second} \),
				\(\phi (t=0) = \frac{\pi}{8}\), \(\dot{\phi} (t=0) = 0\).
	\item \(m_1 = \SI{3}{\kilo\gram}\), \(m_2 = \SI{1}{\kilo\gram}\),
				\(\ell_1 = \SI{1}{\metre}\), \(\ell_2 = \SI{1}{\metre}\),\\ 
				\(\phi_1 (t=0) = \frac{\pi}{8}\), \(\dot{\phi}_1 (t=0) = 0\), 
				\(\phi_2 (t=0) = \frac{\pi}{4}\), \(\dot{\phi}_2 (t=0) = -\frac{\pi}{16} \si{\per\second}\).
\end{enumerate}


\item 
	\begin{minipage}[t][2cm]{0.7\textwidth}
	\textbf{Pendulum of linked beads moving on rigid thin wires}\\
	Time from \(t = \SI{0}{\second}\) to \(t = \SI{10}{\second}\). Parameters and initial conditions:\\
	\(m_1 = m_2 = m = \SI{2}{\kilo\gram}\), \(l = \SI{2}{\metre}\), \(\theta(t=0) = \frac{\pi}{4}\), \(\dot{\theta}(t=0) = 0\).
	\end{minipage}
	\begin{minipage}[c][2cm][t]{0.3\textwidth}
		\hspace{0.5cm}
  	 \includegraphics[width=0.6\textwidth]{figures/fcen1-004}
	\end{minipage}


\item
\begin{minipage}[t][2cm]{0.65\textwidth}
(*) \textbf{Compound Atwood machine} [Marion ex. 7.8]\\
	Time from \(t = \SI{0}{\second}\) to \(t = \SI{5}{\second}\). Parameters and initial conditions:\\
\(\ell_\text{top} = \SI{15}{\metre}\), 
\(R_{\text{top pulley}} = \SI{0.5}{\metre}\), 
\(\ell_\text{bottom} = \SI{15}{\metre}\), 
\(R_{\text{bottom pulley}} = \SI{0.5}{\metre}\),\\ 
\(m_1 = \SI{1}{\kilo\gram}\),
\(m_2 = \SI{2}{\kilo\gram}\),
\(m_3 = \SI{3}{\kilo\gram}\),
\(M_{\text{top pulley}} = \SI{4}{\kilo\gram}\),
\(M_{\text{bottom pulley}} = \SI{4}{\kilo\gram}\),\\
\(y(t=0) = \SI{1}{\metre}\), \(\dot{y}_1(t=0) = 0\),
\(y_2(t=0) = \SI{2}{\metre}\), \(\dot{y}_2(t=0) = 0\)
\end{minipage}
\begin{minipage}[c][3cm][t]{0.3\textwidth}
	\includegraphics[width=\textwidth]{figures/marion_fig7_6}
\end{minipage}

\end{enumerate}
\end{document}
