\documentclass[11pt, english, a4paper, twopage]{article}
% Version en 2024 Víctor Bettachini < vbettachini@unlam.edu.ar >

\usepackage[T1]{fontenc}
\usepackage[utf8]{inputenc}

% \usepackage[spanish, es-tabla]{babel}
% \def\spanishoptions{argentina} % Was macht dass?
% \usepackage{babelbib}
% \selectbiblanguage{spanish}
% \addto\shorthandsspanish{\spanishdeactivate{~<>}}

\usepackage{graphicx}
\graphicspath{{../figuresLaTeX/}}
% \usepackage{float}

\usepackage[arrowdel]{physics}
\newcommand{\pvec}[1]{\vec{#1}\mkern2mu\vphantom{#1}}
% \usepackage{units}
\usepackage[separate-uncertainty= true, multi-part-units= single, range-units= single, range-phrase= {~a~}, locale= FR]{siunitx}
\usepackage{isotope} % $\isotope[A][Z]{X}\to\isotope[A-4][Z-2]{Y}+\isotope[4][2]{\alpha}

\usepackage{tasks}
\usepackage[inline]{enumitem}
% \usepackage{enumerate}

\usepackage{hyperref}

% \usepackage{amsmath}
% \usepackage{amstext}
% \usepackage{amssymb}

\usepackage{tikz}
\usepackage{tikz-3dplot}
\usepackage{tikz-dimline}
\usetikzlibrary{calc}
% \usetikzlibrary{math}
\usetikzlibrary{arrows.meta}
\usetikzlibrary{snakes}
\usetikzlibrary{decorations}
\usetikzlibrary{decorations.pathmorphing}
\usetikzlibrary{patterns}

\usepackage[hmargin=1cm,vmargin=3cm, top= 0.75cm,nohead]{geometry}

\usepackage{lastpage}
\usepackage{fancyhdr}
\pagestyle{fancyplain}
\fancyhf{}
\setlength\headheight{28.7pt} 
\fancyhead[LE, LO]{\textbf{Computational Analytical Mechanics} }
% \fancyhead[LE, LO]{\textbf{Mecánica General} }
\fancyhead[RE, RO]{\href{https://ingenieria.unlam.edu.ar/}{$\vcenter{\hbox{\includegraphics[height=1cm]{ambos.pdf}}}$}}
\fancyfoot{\href{https://creativecommons.org/licenses/by-nc-sa/4.0/}{$\vcenter{\hbox{\includegraphics[height=0.4cm]{by-nc-sa_80x15.pdf}}}$} \href{https://ingenieria.unlam.edu.ar/}{DIIT - UNLaM}}
\fancyfoot[C]{ {\tiny Updated \today} }
\fancyfoot[RO, LE]{Page \thepage/\pageref{LastPage}}
\renewcommand{\headrulewidth}{0pt}
\renewcommand{\footrulewidth}{0pt}

\begin{document}
\begin{center}
 \textsc{\large Oscillations in discrete systems | Multiple degrees of freedom}
\end{center}
\begin{enumerate}
\item
\begin{minipage}[t][4cm]{0.65\textwidth}
 \textbf{Multiple carts}
 In the system shown in the figure \(l = \SI{0.5}{\metre}\), \(k_1 = k_2 = k = \SI{2E3}{\newton\per\metre}\) and \(m_1 = m_2 = m_3 = m = \SI{1}{\kilo\gram}\).
 Assuming small oscillations around zero of the indicated coordinates:
 \begin{tasks}
 \task obtain the Euler-Lagrange equations,
 \task write them in matrix form (matrices M, K), and
 \task obtain the natural oscillation frequencies of the system.
 \end{tasks}
\end{minipage}
\begin{minipage}[c][0cm][t]{0.3\textwidth}
 \includegraphics[width=\textwidth]{figures/shabana_fig_P3_6.png}
\end{minipage}
\item
\begin{minipage}[t][2.5cm]{0.65\textwidth}
 \textbf{Compound torsional pendulum}
 The system in the figure consists of a shaft that passes through three disks having moments of inertia \(I_1, I_2\) and \(I_3\) all of equal magnitude \SI{1E3}{\kilo\gram\metre\squared}.
 The steel shaft has a diameter \(d = \SI{0.01}{\metre}\) and its sections have lengths of \(l_1 = l_2 = l_3 = \SI{0.5}{\metre}\).
\end{minipage}
\begin{minipage}[c][1cm][t]{0.3\textwidth}
 \includegraphics[width=\textwidth]{figures/shabana_fig_P3_5.png}
\end{minipage}
Let us recall that for an angular coordinate \(\theta\) the Euler-Lagrange equation is
\[
 \Gamma \dot{\theta} + \kappa \theta + I \ddot{\theta} = \tau,
\]
where \(\Gamma\) is the torsional friction, \(I\) the moment of inertia, \(\tau\) the applied torque.
$\kappa$ is the \href{https://en.wikipedia.org/wiki/Torsion_spring#Torsion_coefficient}{torsional stiffness or torsion coefficient} that responds to the restoring torque exerted by the piece when twisted through a unit angle, $\tau_\text{restoring} = - \kappa \theta$ and its magnitude is determined by
\[
 \kappa = \frac{G J}{l},
\]
where \(l\) is the length of the piece, \(G\) the \href{https://en.wikipedia.org/wiki/Shear_modulus}{shear modulus} specific to each material, and \(J\) is the \href{https://en.wikipedia.org/wiki/Torsion_constant}{torsion constant} of the cross-sectional geometry transverse to the direction of $\vec{\tau}$.
For a circular section \(J\) is equal to the second moment of area, or \href{https://en.wikipedia.org/wiki/Second_moment_of_area}{polar moment of inertia}
\[
 J_{zz} = J_{xx} + J_{yy} = \frac{\pi r^4}{2} = \frac{\pi d^4}{32}.
\]
According to the document \href{https://tsapps.nist.gov/publication/get_pdf.cfm?pub_id=101021}{Mechanical Properties of Structural Steels} published by the U.S. National Institute of Standards and Technology, for the structural steel of the \href{https://en.wikipedia.org/wiki/World_Trade_Center_(1973%E2%80%932001)}{towers 1 and 2 of the World Trade Center in New York that disappeared in 2001},
\[
 \begin{aligned}
 G &= g_0 + g_1 T + g_2 T^2 + g_3 T^3 + g_4 T^4 + g_5 T^5\\
 g_0 &= \SI{80.005922}{\giga\pascal}\\
 g_1 &= \SI{-0.018303811}{\giga\pascal\per\celsius}\\
 g_2 &= \SI{-1.5650288E-5}{\giga\pascal\per\celsius\tothe{2}}\\
 g_3 &= \SI{-1.5160921E-8}{\giga\pascal\per\celsius\tothe{3}}\\
 g_4 &= \SI{-1.6242911E-11}{\giga\pascal\per\celsius\tothe{4}}\\
 g_5 &= \SI{7.7277543E-15}{\giga\pascal\per\celsius\tothe{5}}
 \end{aligned}
\]
Disregarding rotational friction \(\Gamma\):
\begin{tasks}
 \task obtain the Euler-Lagrange equations,
 \task write them in matrix form (matrices M, K), and
 \task obtain the natural oscillation frequencies of the system.
\end{tasks}
\end{enumerate}
\end{document}